\PassOptionsToPackage{unicode=true}{hyperref} % options for packages loaded elsewhere
\PassOptionsToPackage{hyphens}{url}
%
\documentclass[]{article}
\usepackage{lmodern}
\usepackage{amssymb,amsmath}
\usepackage{ifxetex,ifluatex}
\usepackage{fixltx2e} % provides \textsubscript
\ifnum 0\ifxetex 1\fi\ifluatex 1\fi=0 % if pdftex
  \usepackage[T1]{fontenc}
  \usepackage[utf8]{inputenc}
  \usepackage{textcomp} % provides euro and other symbols
\else % if luatex or xelatex
  \usepackage{unicode-math}
  \defaultfontfeatures{Ligatures=TeX,Scale=MatchLowercase}
\fi
% use upquote if available, for straight quotes in verbatim environments
\IfFileExists{upquote.sty}{\usepackage{upquote}}{}
% use microtype if available
\IfFileExists{microtype.sty}{%
\usepackage[]{microtype}
\UseMicrotypeSet[protrusion]{basicmath} % disable protrusion for tt fonts
}{}
\IfFileExists{parskip.sty}{%
\usepackage{parskip}
}{% else
\setlength{\parindent}{0pt}
\setlength{\parskip}{6pt plus 2pt minus 1pt}
}
\usepackage{hyperref}
\hypersetup{
            pdftitle={part 3},
            pdfborder={0 0 0},
            breaklinks=true}
\urlstyle{same}  % don't use monospace font for urls
\usepackage[margin=1in]{geometry}
\usepackage{color}
\usepackage{fancyvrb}
\newcommand{\VerbBar}{|}
\newcommand{\VERB}{\Verb[commandchars=\\\{\}]}
\DefineVerbatimEnvironment{Highlighting}{Verbatim}{commandchars=\\\{\}}
% Add ',fontsize=\small' for more characters per line
\usepackage{framed}
\definecolor{shadecolor}{RGB}{248,248,248}
\newenvironment{Shaded}{\begin{snugshade}}{\end{snugshade}}
\newcommand{\AlertTok}[1]{\textcolor[rgb]{0.94,0.16,0.16}{#1}}
\newcommand{\AnnotationTok}[1]{\textcolor[rgb]{0.56,0.35,0.01}{\textbf{\textit{#1}}}}
\newcommand{\AttributeTok}[1]{\textcolor[rgb]{0.77,0.63,0.00}{#1}}
\newcommand{\BaseNTok}[1]{\textcolor[rgb]{0.00,0.00,0.81}{#1}}
\newcommand{\BuiltInTok}[1]{#1}
\newcommand{\CharTok}[1]{\textcolor[rgb]{0.31,0.60,0.02}{#1}}
\newcommand{\CommentTok}[1]{\textcolor[rgb]{0.56,0.35,0.01}{\textit{#1}}}
\newcommand{\CommentVarTok}[1]{\textcolor[rgb]{0.56,0.35,0.01}{\textbf{\textit{#1}}}}
\newcommand{\ConstantTok}[1]{\textcolor[rgb]{0.00,0.00,0.00}{#1}}
\newcommand{\ControlFlowTok}[1]{\textcolor[rgb]{0.13,0.29,0.53}{\textbf{#1}}}
\newcommand{\DataTypeTok}[1]{\textcolor[rgb]{0.13,0.29,0.53}{#1}}
\newcommand{\DecValTok}[1]{\textcolor[rgb]{0.00,0.00,0.81}{#1}}
\newcommand{\DocumentationTok}[1]{\textcolor[rgb]{0.56,0.35,0.01}{\textbf{\textit{#1}}}}
\newcommand{\ErrorTok}[1]{\textcolor[rgb]{0.64,0.00,0.00}{\textbf{#1}}}
\newcommand{\ExtensionTok}[1]{#1}
\newcommand{\FloatTok}[1]{\textcolor[rgb]{0.00,0.00,0.81}{#1}}
\newcommand{\FunctionTok}[1]{\textcolor[rgb]{0.00,0.00,0.00}{#1}}
\newcommand{\ImportTok}[1]{#1}
\newcommand{\InformationTok}[1]{\textcolor[rgb]{0.56,0.35,0.01}{\textbf{\textit{#1}}}}
\newcommand{\KeywordTok}[1]{\textcolor[rgb]{0.13,0.29,0.53}{\textbf{#1}}}
\newcommand{\NormalTok}[1]{#1}
\newcommand{\OperatorTok}[1]{\textcolor[rgb]{0.81,0.36,0.00}{\textbf{#1}}}
\newcommand{\OtherTok}[1]{\textcolor[rgb]{0.56,0.35,0.01}{#1}}
\newcommand{\PreprocessorTok}[1]{\textcolor[rgb]{0.56,0.35,0.01}{\textit{#1}}}
\newcommand{\RegionMarkerTok}[1]{#1}
\newcommand{\SpecialCharTok}[1]{\textcolor[rgb]{0.00,0.00,0.00}{#1}}
\newcommand{\SpecialStringTok}[1]{\textcolor[rgb]{0.31,0.60,0.02}{#1}}
\newcommand{\StringTok}[1]{\textcolor[rgb]{0.31,0.60,0.02}{#1}}
\newcommand{\VariableTok}[1]{\textcolor[rgb]{0.00,0.00,0.00}{#1}}
\newcommand{\VerbatimStringTok}[1]{\textcolor[rgb]{0.31,0.60,0.02}{#1}}
\newcommand{\WarningTok}[1]{\textcolor[rgb]{0.56,0.35,0.01}{\textbf{\textit{#1}}}}
\usepackage{graphicx,grffile}
\makeatletter
\def\maxwidth{\ifdim\Gin@nat@width>\linewidth\linewidth\else\Gin@nat@width\fi}
\def\maxheight{\ifdim\Gin@nat@height>\textheight\textheight\else\Gin@nat@height\fi}
\makeatother
% Scale images if necessary, so that they will not overflow the page
% margins by default, and it is still possible to overwrite the defaults
% using explicit options in \includegraphics[width, height, ...]{}
\setkeys{Gin}{width=\maxwidth,height=\maxheight,keepaspectratio}
\setlength{\emergencystretch}{3em}  % prevent overfull lines
\providecommand{\tightlist}{%
  \setlength{\itemsep}{0pt}\setlength{\parskip}{0pt}}
\setcounter{secnumdepth}{0}
% Redefines (sub)paragraphs to behave more like sections
\ifx\paragraph\undefined\else
\let\oldparagraph\paragraph
\renewcommand{\paragraph}[1]{\oldparagraph{#1}\mbox{}}
\fi
\ifx\subparagraph\undefined\else
\let\oldsubparagraph\subparagraph
\renewcommand{\subparagraph}[1]{\oldsubparagraph{#1}\mbox{}}
\fi

% set default figure placement to htbp
\makeatletter
\def\fps@figure{htbp}
\makeatother


\title{part 3}
\author{}
\date{\vspace{-2.5em}}

\begin{document}
\maketitle

Loading data and fitting models.

\begin{Shaded}
\begin{Highlighting}[]
\KeywordTok{load}\NormalTok{(}\StringTok{'weather.rda'}\NormalTok{) }\CommentTok{# load data }
\KeywordTok{library}\NormalTok{(ggplot2)}
\NormalTok{model}\FloatTok{.1}\NormalTok{a <-}\StringTok{ }\KeywordTok{lm}\NormalTok{(rain }\OperatorTok{~}\StringTok{ }\NormalTok{temp, }\DataTypeTok{data =}\NormalTok{ weather)}
\NormalTok{model}\FloatTok{.1}\NormalTok{b <-}\StringTok{ }\KeywordTok{lm}\NormalTok{(}\KeywordTok{log}\NormalTok{(rain) }\OperatorTok{~}\StringTok{ }\NormalTok{temp, }\DataTypeTok{data =}\NormalTok{ weather)}
\NormalTok{model}\FloatTok{.2}\NormalTok{c <-}\StringTok{ }\KeywordTok{lm}\NormalTok{(}\KeywordTok{log}\NormalTok{(rain) }\OperatorTok{~}\StringTok{ }\NormalTok{temp }\OperatorTok{+}\StringTok{ }\NormalTok{pressure, }\DataTypeTok{data =}\NormalTok{ weather)}
\NormalTok{model}\FloatTok{.2}\NormalTok{h <-}\StringTok{ }\KeywordTok{lm}\NormalTok{(}\KeywordTok{log}\NormalTok{(rain)}\OperatorTok{~}\NormalTok{temp}\OperatorTok{*}\NormalTok{pressure, }\DataTypeTok{data =}\NormalTok{ weather)}
\NormalTok{weather}\OperatorTok{$}\NormalTok{location <-}\StringTok{ }\KeywordTok{relevel}\NormalTok{(weather}\OperatorTok{$}\NormalTok{location, }\StringTok{"Uppsala"}\NormalTok{)}
\NormalTok{model}\FloatTok{.2}\NormalTok{n <-}\StringTok{ }\KeywordTok{lm}\NormalTok{(}\KeywordTok{log}\NormalTok{(rain) }\OperatorTok{~}\StringTok{ }\NormalTok{temp}\OperatorTok{*}\NormalTok{pressure }\OperatorTok{+}\StringTok{ }\NormalTok{location, }\DataTypeTok{data =}\NormalTok{ weather)}
\end{Highlighting}
\end{Shaded}

\hypertarget{outliers-and-influential-observations}{%
\section{3.1 Outliers and influential
observations}\label{outliers-and-influential-observations}}

Since the leverage is a measure of how much an observation is an outlier
with respect to the feature space, it will be higher for points with
``unusual'' features. This means the leverages will, in general, be
higher for more unusual locations.

\begin{Shaded}
\begin{Highlighting}[]
\NormalTok{model}\FloatTok{.2}\NormalTok{n}\OperatorTok{$}\NormalTok{leverages <-}\StringTok{ }\KeywordTok{influence}\NormalTok{(model}\FloatTok{.2}\NormalTok{n)}\OperatorTok{$}\NormalTok{hat }
\NormalTok{weather}\OperatorTok{$}\NormalTok{leverage <-}\StringTok{ }\NormalTok{model}\FloatTok{.2}\NormalTok{n}\OperatorTok{$}\NormalTok{leverages}
\NormalTok{n =}\StringTok{ }\KeywordTok{nrow}\NormalTok{(weather)}
\NormalTok{p =}\StringTok{ }\KeywordTok{length}\NormalTok{(model}\FloatTok{.2}\NormalTok{n}\OperatorTok{$}\NormalTok{coefficients)}
\NormalTok{plot}\FloatTok{.3}\NormalTok{a <-}\StringTok{ }\KeywordTok{ggplot}\NormalTok{(}\DataTypeTok{data =}\NormalTok{ weather, }\KeywordTok{aes}\NormalTok{(}\DataTypeTok{y =}\NormalTok{ leverage)) }\OperatorTok{+}\StringTok{ }
\StringTok{  }\KeywordTok{geom_hline}\NormalTok{(}\DataTypeTok{yintercept =} \DecValTok{1}\OperatorTok{/}\NormalTok{n, }\DataTypeTok{color =} \StringTok{"red"}\NormalTok{, }\DataTypeTok{linetype =} \StringTok{"dashed"}\NormalTok{) }\OperatorTok{+}\StringTok{ }
\StringTok{  }\KeywordTok{geom_hline}\NormalTok{(}\DataTypeTok{yintercept =} \DecValTok{2}\OperatorTok{*}\NormalTok{(p}\OperatorTok{+}\DecValTok{1}\NormalTok{)}\OperatorTok{/}\NormalTok{n, }\DataTypeTok{color =} \StringTok{"red"}\NormalTok{) }\OperatorTok{+}\StringTok{ }
\StringTok{  }\KeywordTok{geom_hline}\NormalTok{(}\DataTypeTok{yintercept =} \FloatTok{0.026}\NormalTok{) }\OperatorTok{+}\StringTok{ }
\StringTok{  }\KeywordTok{facet_wrap}\NormalTok{(}\OperatorTok{~}\NormalTok{location)}

\NormalTok{plot}\FloatTok{.3}\NormalTok{a.temp <-}\StringTok{ }\NormalTok{plot}\FloatTok{.3}\NormalTok{a }\OperatorTok{+}
\StringTok{  }\KeywordTok{geom_point}\NormalTok{(}\KeywordTok{aes}\NormalTok{(}\DataTypeTok{x =}\NormalTok{ temp))}
\NormalTok{plot}\FloatTok{.3}\NormalTok{a.pressure <-}\StringTok{ }\NormalTok{plot}\FloatTok{.3}\NormalTok{a }\OperatorTok{+}\StringTok{ }
\StringTok{  }\KeywordTok{geom_point}\NormalTok{(}\KeywordTok{aes}\NormalTok{(}\DataTypeTok{x =}\NormalTok{ pressure))}

\NormalTok{plot}\FloatTok{.3}\NormalTok{a.temp}
\end{Highlighting}
\end{Shaded}

\includegraphics{part3_files/figure-latex/unnamed-chunk-27-1.pdf}

\begin{Shaded}
\begin{Highlighting}[]
\NormalTok{plot}\FloatTok{.3}\NormalTok{a.pressure}
\end{Highlighting}
\end{Shaded}

\includegraphics{part3_files/figure-latex/unnamed-chunk-27-2.pdf}

\hypertarget{b}{%
\subsection{3 b)}\label{b}}

The leverage is high since they are have are outliers in the
temp-pressure plane. None of the points are really outliers in
temperature (can bee seen by projecting points on the y axis), while
some of them are outliers in pressures and some not (project points ot x
axis).

\begin{Shaded}
\begin{Highlighting}[]
\NormalTok{problematic_indices =}\StringTok{ }\KeywordTok{which}\NormalTok{(weather}\OperatorTok{$}\NormalTok{leverage }\OperatorTok{>}\StringTok{ }\FloatTok{0.026}\NormalTok{ )}
\NormalTok{plot}\FloatTok{.3}\NormalTok{b <-}\StringTok{ }\KeywordTok{ggplot}\NormalTok{(weather, }\KeywordTok{aes}\NormalTok{(}\DataTypeTok{x =}\NormalTok{ pressure, }\DataTypeTok{y =}\NormalTok{ temp)) }\OperatorTok{+}\StringTok{ }
\StringTok{  }\KeywordTok{geom_point}\NormalTok{() }\OperatorTok{+}\StringTok{ }
\StringTok{  }\KeywordTok{geom_point}\NormalTok{(}\DataTypeTok{data =}\NormalTok{ weather[problematic_indices, ], }\DataTypeTok{color =} \StringTok{"red"}\NormalTok{) }\OperatorTok{+}\StringTok{ }
\StringTok{  }\KeywordTok{facet_wrap}\NormalTok{(}\OperatorTok{~}\NormalTok{location)}
\NormalTok{plot}\FloatTok{.3}\NormalTok{b}
\end{Highlighting}
\end{Shaded}

\includegraphics{part3_files/figure-latex/unnamed-chunk-28-1.pdf}

\hypertarget{c}{%
\subsection{3 c)}\label{c}}

\begin{Shaded}
\begin{Highlighting}[]
\NormalTok{model}\FloatTok{.2}\NormalTok{n.pred <-}\StringTok{ }\NormalTok{pred}\FloatTok{.8}\NormalTok{c <-}\StringTok{ }\KeywordTok{cbind}\NormalTok{(weather, }
                 \DataTypeTok{conf =}\NormalTok{ (}\KeywordTok{predict}\NormalTok{(model}\FloatTok{.2}\NormalTok{n, }\DataTypeTok{interval =} \StringTok{"confidence"}\NormalTok{)),}
                 \DataTypeTok{pred =}\NormalTok{ (}\KeywordTok{predict}\NormalTok{(model}\FloatTok{.2}\NormalTok{n, }\DataTypeTok{interval =} \StringTok{"prediction"}\NormalTok{))) }
\end{Highlighting}
\end{Shaded}

\begin{verbatim}
## Warning in predict.lm(model.2n, interval = "prediction"): predictions on current data refer to _future_ responses
\end{verbatim}

\begin{Shaded}
\begin{Highlighting}[]
\NormalTok{model}\FloatTok{.2}\NormalTok{n.pred}\OperatorTok{$}\NormalTok{stud.res <-}\StringTok{ }\KeywordTok{rstudent}\NormalTok{(model}\FloatTok{.2}\NormalTok{n)}
\NormalTok{plot}\FloatTok{.3}\NormalTok{c <-}\StringTok{ }\KeywordTok{ggplot}\NormalTok{(}\DataTypeTok{data =}\NormalTok{ model}\FloatTok{.2}\NormalTok{n.pred, }\KeywordTok{aes}\NormalTok{(}\DataTypeTok{x =}\NormalTok{ pred.fit, }\DataTypeTok{y =}\NormalTok{ (stud.res)))}\OperatorTok{+}
\StringTok{  }\KeywordTok{geom_point}\NormalTok{() }\OperatorTok{+}\StringTok{ }
\StringTok{  }\KeywordTok{geom_point}\NormalTok{(}\DataTypeTok{data =}\NormalTok{ model}\FloatTok{.2}\NormalTok{n.pred[problematic_indices,], }\DataTypeTok{color =} \StringTok{"red"}\NormalTok{) }\OperatorTok{+}\StringTok{ }
\StringTok{  }\KeywordTok{facet_wrap}\NormalTok{(}\OperatorTok{~}\NormalTok{location)}
\NormalTok{plot}\FloatTok{.3}\NormalTok{c}
\end{Highlighting}
\end{Shaded}

\includegraphics{part3_files/figure-latex/unnamed-chunk-29-1.pdf}

\hypertarget{d}{%
\subsection{3 d)}\label{d}}

\begin{Shaded}
\begin{Highlighting}[]
\NormalTok{uppsala_index =}\StringTok{ }\KeywordTok{which}\NormalTok{(}\KeywordTok{abs}\NormalTok{(model}\FloatTok{.2}\NormalTok{n.pred}\OperatorTok{$}\NormalTok{stud.res) }\OperatorTok{>}\StringTok{ }\DecValTok{8}\NormalTok{)}
\NormalTok{abisko_index =}\StringTok{ }\KeywordTok{which}\NormalTok{(}\KeywordTok{abs}\NormalTok{(model}\FloatTok{.2}\NormalTok{n.pred}\OperatorTok{$}\NormalTok{stud.res) }\OperatorTok{>}\StringTok{ }\DecValTok{3} \OperatorTok{&}\StringTok{ }\NormalTok{model}\FloatTok{.2}\NormalTok{n.pred}\OperatorTok{$}\NormalTok{location }\OperatorTok{==}\StringTok{ "Abisko"}\NormalTok{)}
\NormalTok{lund_index =}\StringTok{ }\KeywordTok{which}\NormalTok{(}\KeywordTok{abs}\NormalTok{(model}\FloatTok{.2}\NormalTok{n.pred}\OperatorTok{$}\NormalTok{stud.res) }\OperatorTok{>}\StringTok{ }\DecValTok{3} \OperatorTok{&}\StringTok{ }\NormalTok{model}\FloatTok{.2}\NormalTok{n.pred}\OperatorTok{$}\NormalTok{location }\OperatorTok{==}\StringTok{ "Lund"}\NormalTok{)}
\NormalTok{plot}\FloatTok{.3}\NormalTok{c.temp <-}\StringTok{ }\KeywordTok{ggplot}\NormalTok{(}\DataTypeTok{data =}\NormalTok{ model}\FloatTok{.2}\NormalTok{n.pred, }\KeywordTok{aes}\NormalTok{(}\DataTypeTok{y =} \KeywordTok{log}\NormalTok{(rain), }\DataTypeTok{x =}\NormalTok{ temp)) }\OperatorTok{+}\StringTok{ }
\StringTok{  }\KeywordTok{geom_point}\NormalTok{() }\OperatorTok{+}\StringTok{ }
\StringTok{  }\KeywordTok{geom_point}\NormalTok{(}\DataTypeTok{data =}\NormalTok{ model}\FloatTok{.2}\NormalTok{n.pred[problematic_indices,], }\DataTypeTok{color =} \StringTok{"red"}\NormalTok{)}\OperatorTok{+}
\StringTok{  }\KeywordTok{facet_wrap}\NormalTok{(}\OperatorTok{~}\NormalTok{location) }\OperatorTok{+}\StringTok{ }
\StringTok{  }\KeywordTok{geom_point}\NormalTok{(}\DataTypeTok{data =}\NormalTok{ model}\FloatTok{.2}\NormalTok{n.pred[}\KeywordTok{c}\NormalTok{(uppsala_index, abisko_index, lund_index),], }\DataTypeTok{color =} \StringTok{"green"}\NormalTok{)}
\NormalTok{plot}\FloatTok{.3}\NormalTok{c.temp}
\end{Highlighting}
\end{Shaded}

\includegraphics{part3_files/figure-latex/unnamed-chunk-30-1.pdf}

\begin{Shaded}
\begin{Highlighting}[]
\NormalTok{plot}\FloatTok{.3}\NormalTok{c.pressure <-}\KeywordTok{ggplot}\NormalTok{(}\DataTypeTok{data =}\NormalTok{ model}\FloatTok{.2}\NormalTok{n.pred, }\KeywordTok{aes}\NormalTok{(}\DataTypeTok{y =} \KeywordTok{log}\NormalTok{(rain), }\DataTypeTok{x =}\NormalTok{ pressure)) }\OperatorTok{+}\StringTok{ }
\StringTok{  }\KeywordTok{geom_point}\NormalTok{() }\OperatorTok{+}\StringTok{ }
\StringTok{  }\KeywordTok{geom_point}\NormalTok{(}\DataTypeTok{data =}\NormalTok{ model}\FloatTok{.2}\NormalTok{n.pred[problematic_indices,], }\DataTypeTok{color =} \StringTok{"red"}\NormalTok{)}\OperatorTok{+}
\StringTok{  }\KeywordTok{facet_wrap}\NormalTok{(}\OperatorTok{~}\NormalTok{location) }\OperatorTok{+}\StringTok{ }
\StringTok{  }\KeywordTok{geom_point}\NormalTok{(}\DataTypeTok{data =}\NormalTok{ model}\FloatTok{.2}\NormalTok{n.pred[}\KeywordTok{c}\NormalTok{(uppsala_index, abisko_index, lund_index),], }\DataTypeTok{color =} \StringTok{"green"}\NormalTok{)}

\NormalTok{plot}\FloatTok{.3}\NormalTok{c.pressure}
\end{Highlighting}
\end{Shaded}

\includegraphics{part3_files/figure-latex/unnamed-chunk-30-2.pdf}

\hypertarget{e}{%
\subsection{3 e)}\label{e}}

\begin{Shaded}
\begin{Highlighting}[]
\NormalTok{model}\FloatTok{.2}\NormalTok{n.pred}\OperatorTok{$}\NormalTok{D <-}\StringTok{ }\KeywordTok{cooks.distance}\NormalTok{(model}\FloatTok{.2}\NormalTok{n)}
\NormalTok{plot}\FloatTok{.3}\NormalTok{e <-}\StringTok{ }\KeywordTok{ggplot}\NormalTok{(model}\FloatTok{.2}\NormalTok{n.pred, }\KeywordTok{aes}\NormalTok{(}\DataTypeTok{x =}\NormalTok{ temp, }\DataTypeTok{y =}\NormalTok{ D)) }\OperatorTok{+}\StringTok{ }
\StringTok{  }\KeywordTok{geom_point}\NormalTok{() }\OperatorTok{+}\StringTok{ }
\StringTok{  }\KeywordTok{geom_point}\NormalTok{(}\DataTypeTok{data =}\NormalTok{ model}\FloatTok{.2}\NormalTok{n.pred[problematic_indices, ], }\DataTypeTok{color =} \StringTok{"red"}\NormalTok{) }\OperatorTok{+}
\StringTok{  }\KeywordTok{geom_point}\NormalTok{(}\DataTypeTok{data =}\NormalTok{ model}\FloatTok{.2}\NormalTok{n.pred[}\KeywordTok{c}\NormalTok{(uppsala_index, abisko_index, lund_index),], }\DataTypeTok{color =} \StringTok{"green"}\NormalTok{)}
\NormalTok{plot}\FloatTok{.3}\NormalTok{e}
\end{Highlighting}
\end{Shaded}

\includegraphics{part3_files/figure-latex/unnamed-chunk-31-1.pdf}

\hypertarget{f}{%
\subsection{3f)}\label{f}}

\begin{Shaded}
\begin{Highlighting}[]
\NormalTok{large_leverage_points =}\StringTok{ }\NormalTok{problematic_indices}
\NormalTok{large_residual_points =}\StringTok{ }\KeywordTok{c}\NormalTok{(uppsala_index, abisko_index, lund_index)}
\NormalTok{large_D_points =}\StringTok{ }\KeywordTok{which}\NormalTok{(model}\FloatTok{.2}\NormalTok{n.pred}\OperatorTok{$}\NormalTok{D }\OperatorTok{>}\StringTok{ }\FloatTok{0.02}\NormalTok{) }\CommentTok{# the 5 largest ones}
\NormalTok{weather.clean =}\StringTok{ }\NormalTok{weather[}\OperatorTok{-}\KeywordTok{c}\NormalTok{(large_residual_points, large_D_points),] }\CommentTok{#remove 5 points with highest Cook's D, and the two other with large residuals }
\NormalTok{weather.clean}\OperatorTok{$}\NormalTok{location <-}\StringTok{ }\KeywordTok{relevel}\NormalTok{(weather.clean}\OperatorTok{$}\NormalTok{location, }\StringTok{"Uppsala"}\NormalTok{)}
\NormalTok{model}\FloatTok{.3}\NormalTok{f <-}\StringTok{ }\KeywordTok{lm}\NormalTok{(}\KeywordTok{log}\NormalTok{(rain) }\OperatorTok{~}\StringTok{ }\NormalTok{temp}\OperatorTok{*}\NormalTok{pressure }\OperatorTok{+}\StringTok{ }\NormalTok{location, }\DataTypeTok{data =}\NormalTok{ weather.clean)}
\NormalTok{weather.clean}\OperatorTok{$}\NormalTok{stud.res <-}\StringTok{ }\KeywordTok{rstudent}\NormalTok{(model}\FloatTok{.3}\NormalTok{f)}
\NormalTok{weather.clean}\OperatorTok{$}\NormalTok{D <-}\StringTok{ }\KeywordTok{cooks.distance}\NormalTok{(model}\FloatTok{.3}\NormalTok{f)}
\NormalTok{plot}\FloatTok{.3}\NormalTok{f.D <-}\StringTok{ }\KeywordTok{ggplot}\NormalTok{(}\DataTypeTok{data=}\NormalTok{ weather.clean, }\KeywordTok{aes}\NormalTok{(}\DataTypeTok{x =}\NormalTok{ temp,}\DataTypeTok{y =}\NormalTok{ D)) }\OperatorTok{+}\StringTok{ }
\StringTok{  }\KeywordTok{geom_point}\NormalTok{()}
\NormalTok{plot}\FloatTok{.3}\NormalTok{f.res <-}\StringTok{ }\KeywordTok{ggplot}\NormalTok{(}\DataTypeTok{data =}\NormalTok{ weather.clean, }\KeywordTok{aes}\NormalTok{(}\DataTypeTok{x =}\NormalTok{ temp, }\DataTypeTok{y =}\NormalTok{ stud.res)) }\OperatorTok{+}\StringTok{ }
\StringTok{  }\KeywordTok{geom_point}\NormalTok{()}
\NormalTok{plot}\FloatTok{.3}\NormalTok{f.D}
\end{Highlighting}
\end{Shaded}

\includegraphics{part3_files/figure-latex/unnamed-chunk-32-1.pdf}

\begin{Shaded}
\begin{Highlighting}[]
\NormalTok{plot}\FloatTok{.3}\NormalTok{f.res}
\end{Highlighting}
\end{Shaded}

\includegraphics{part3_files/figure-latex/unnamed-chunk-32-2.pdf}

\hypertarget{model-comparisons}{%
\section{3.2 Model comparisons}\label{model-comparisons}}

\hypertarget{g}{%
\subsection{3g)}\label{g}}

last model (2n) is best, and it explains \textasciitilde{} 40\% of the
variability in the data

\begin{Shaded}
\begin{Highlighting}[]
\NormalTok{model}\FloatTok{.1}\NormalTok{b.clean <-}\StringTok{ }\KeywordTok{lm}\NormalTok{(}\KeywordTok{log}\NormalTok{(rain) }\OperatorTok{~}\StringTok{ }\NormalTok{temp, }\DataTypeTok{data =}\NormalTok{ weather)}
\NormalTok{model}\FloatTok{.2}\NormalTok{c.clean <-}\StringTok{ }\KeywordTok{lm}\NormalTok{(}\KeywordTok{log}\NormalTok{(rain) }\OperatorTok{~}\StringTok{ }\NormalTok{temp }\OperatorTok{+}\StringTok{ }\NormalTok{pressure, }\DataTypeTok{data =}\NormalTok{ weather)}
\NormalTok{model}\FloatTok{.2}\NormalTok{h.clean <-}\StringTok{ }\KeywordTok{lm}\NormalTok{(}\KeywordTok{log}\NormalTok{(rain)}\OperatorTok{~}\NormalTok{temp}\OperatorTok{*}\NormalTok{pressure, }\DataTypeTok{data =}\NormalTok{ weather)}
\NormalTok{model}\FloatTok{.2}\NormalTok{n.clean <-}\StringTok{ }\NormalTok{model}\FloatTok{.3}\NormalTok{f}
\NormalTok{R2<-}\StringTok{ }\KeywordTok{c}\NormalTok{(}\KeywordTok{summary}\NormalTok{(model}\FloatTok{.1}\NormalTok{b.clean)}\OperatorTok{$}\NormalTok{r.squared, }\KeywordTok{summary}\NormalTok{(model}\FloatTok{.2}\NormalTok{c.clean)}\OperatorTok{$}\NormalTok{r.squared, }\KeywordTok{summary}\NormalTok{(model}\FloatTok{.2}\NormalTok{h.clean)}\OperatorTok{$}\NormalTok{r.squared, }\KeywordTok{summary}\NormalTok{(model}\FloatTok{.2}\NormalTok{n.clean)}\OperatorTok{$}\NormalTok{r.squared)}
\NormalTok{R2_adj<-}\StringTok{ }\KeywordTok{c}\NormalTok{(}\KeywordTok{summary}\NormalTok{(model}\FloatTok{.1}\NormalTok{b.clean)}\OperatorTok{$}\NormalTok{adj.r.squared, }\KeywordTok{summary}\NormalTok{(model}\FloatTok{.2}\NormalTok{c.clean)}\OperatorTok{$}\NormalTok{adj.r.squared, }\KeywordTok{summary}\NormalTok{(model}\FloatTok{.2}\NormalTok{h.clean)}\OperatorTok{$}\NormalTok{adj.r.squared, }\KeywordTok{summary}\NormalTok{(model}\FloatTok{.2}\NormalTok{n.clean)}\OperatorTok{$}\NormalTok{adj.r.squared)}

\NormalTok{(measures <-}\StringTok{ }\KeywordTok{data.frame}\NormalTok{(}
\NormalTok{                        R2, }
\NormalTok{                        R2_adj,}
                        \KeywordTok{AIC}\NormalTok{(model}\FloatTok{.1}\NormalTok{b.clean, model}\FloatTok{.2}\NormalTok{c.clean,model}\FloatTok{.2}\NormalTok{h.clean, model}\FloatTok{.2}\NormalTok{n.clean)[}\DecValTok{2}\NormalTok{],}
                        \KeywordTok{BIC}\NormalTok{(model}\FloatTok{.1}\NormalTok{b.clean, model}\FloatTok{.2}\NormalTok{c.clean,model}\FloatTok{.2}\NormalTok{h.clean, model}\FloatTok{.2}\NormalTok{n.clean)[}\DecValTok{2}\NormalTok{]))}
\end{Highlighting}
\end{Shaded}

\begin{verbatim}
## Warning in AIC.default(model.1b.clean, model.2c.clean,
## model.2h.clean, model.2n.clean): models are not all fitted
## to the same number of observations
\end{verbatim}

\begin{verbatim}
## Warning in BIC.default(model.1b.clean, model.2c.clean,
## model.2h.clean, model.2n.clean): models are not all fitted
## to the same number of observations
\end{verbatim}

\hypertarget{h}{%
\subsection{3h)}\label{h}}

p-value for partial f-test is almost 0.3 i.e.~no significant improvement

\begin{Shaded}
\begin{Highlighting}[]
\NormalTok{model}\FloatTok{.3}\NormalTok{h.clean <-}\StringTok{ }\KeywordTok{lm}\NormalTok{ (}\KeywordTok{log}\NormalTok{(rain)}\OperatorTok{~}\NormalTok{temp}\OperatorTok{*}\NormalTok{location}\OperatorTok{*}\NormalTok{pressure, weather.clean)}
\KeywordTok{anova}\NormalTok{(model}\FloatTok{.2}\NormalTok{n.clean, model}\FloatTok{.3}\NormalTok{h.clean)}
\end{Highlighting}
\end{Shaded}

\begin{verbatim}
## Analysis of Variance Table
## 
## Model 1: log(rain) ~ temp * pressure + location
## Model 2: log(rain) ~ temp * location * pressure
##   Res.Df    RSS Df Sum of Sq      F Pr(>F)
## 1   1078 356.44                           
## 2   1072 354.04  6    2.3959 1.2091 0.2989
\end{verbatim}

\hypertarget{i}{%
\subsection{3 i)}\label{i}}

The backward elimination stops the best of our previous models
i.e.~model 2n).

\begin{Shaded}
\begin{Highlighting}[]
\KeywordTok{step}\NormalTok{(model}\FloatTok{.3}\NormalTok{h.clean, }\DataTypeTok{k =} \KeywordTok{log}\NormalTok{(}\KeywordTok{nrow}\NormalTok{(weather.clean)))}
\end{Highlighting}
\end{Shaded}

\begin{verbatim}
## Start:  AIC=-1129.13
## log(rain) ~ temp * location * pressure
## 
##                          Df Sum of Sq    RSS     AIC
## - temp:location:pressure  2    1.7934 355.83 -1137.6
## <none>                                354.04 -1129.1
## 
## Step:  AIC=-1137.63
## log(rain) ~ temp + location + pressure + temp:location + temp:pressure + 
##     location:pressure
## 
##                     Df Sum of Sq    RSS     AIC
## - location:pressure  2    0.2584 356.09 -1150.8
## - temp:location      2    0.3591 356.19 -1150.5
## <none>                           355.83 -1137.6
## - temp:pressure      1   13.1818 369.02 -1105.2
## 
## Step:  AIC=-1150.82
## log(rain) ~ temp + location + pressure + temp:location + temp:pressure
## 
##                 Df Sum of Sq    RSS     AIC
## - temp:location  2    0.3441 356.44 -1163.8
## <none>                       356.09 -1150.8
## - temp:pressure  1   16.1064 372.20 -1109.9
## 
## Step:  AIC=-1163.75
## log(rain) ~ temp + location + pressure + temp:pressure
## 
##                 Df Sum of Sq    RSS     AIC
## <none>                       356.44 -1163.8
## - temp:pressure  1    18.492 374.93 -1115.9
## - location       2    50.743 407.18 -1033.5
\end{verbatim}

\begin{verbatim}
## 
## Call:
## lm(formula = log(rain) ~ temp + location + pressure + temp:pressure, 
##     data = weather.clean)
## 
## Coefficients:
##    (Intercept)            temp    locationLund  
##      65.475357        3.398008        0.343368  
## locationAbisko        pressure   temp:pressure  
##      -0.483704       -0.061323       -0.003326
\end{verbatim}

\hypertarget{j}{%
\subsection{3 j)}\label{j}}

arrive at the same model, adding pressure as the first variable.

\begin{Shaded}
\begin{Highlighting}[]
\KeywordTok{step}\NormalTok{(}\KeywordTok{lm}\NormalTok{(}\KeywordTok{log}\NormalTok{(rain)}\OperatorTok{~}\DecValTok{1}\NormalTok{, }\DataTypeTok{data =}\NormalTok{ weather.clean), }
     \DataTypeTok{scope =} \KeywordTok{list}\NormalTok{(}\DataTypeTok{upper =}\NormalTok{ model}\FloatTok{.3}\NormalTok{h.clean), }
     \DataTypeTok{direction =} \StringTok{"forward"}\NormalTok{, }\DataTypeTok{k =} \KeywordTok{log}\NormalTok{(}\KeywordTok{nrow}\NormalTok{(weather.clean)))}
\end{Highlighting}
\end{Shaded}

\begin{verbatim}
## Start:  AIC=-648.5
## log(rain) ~ 1
## 
##            Df Sum of Sq    RSS     AIC
## + pressure  1    72.929 519.20 -783.99
## + temp      1    61.245 530.88 -759.87
## + location  2    49.526 542.60 -729.21
## <none>                  592.13 -648.50
## 
## Step:  AIC=-783.99
## log(rain) ~ pressure
## 
##            Df Sum of Sq    RSS     AIC
## + temp      1    89.943 429.25 -983.22
## + location  2    90.522 428.67 -977.69
## <none>                  519.20 -783.99
## 
## Step:  AIC=-983.22
## log(rain) ~ pressure + temp
## 
##                 Df Sum of Sq    RSS      AIC
## + location       2    54.324 374.93 -1115.92
## + temp:pressure  1    22.073 407.18 -1033.45
## <none>                       429.25  -983.22
## 
## Step:  AIC=-1115.92
## log(rain) ~ pressure + temp + location
## 
##                     Df Sum of Sq    RSS     AIC
## + temp:pressure      1   18.4915 356.44 -1163.8
## <none>                           374.93 -1115.9
## + location:pressure  2    4.5080 370.42 -1115.0
## + temp:location      2    2.7292 372.20 -1109.9
## 
## Step:  AIC=-1163.75
## log(rain) ~ pressure + temp + location + pressure:temp
## 
##                     Df Sum of Sq    RSS     AIC
## <none>                           356.44 -1163.8
## + temp:location      2   0.34413 356.09 -1150.8
## + location:pressure  2   0.24349 356.19 -1150.5
\end{verbatim}

\begin{verbatim}
## 
## Call:
## lm(formula = log(rain) ~ pressure + temp + location + pressure:temp, 
##     data = weather.clean)
## 
## Coefficients:
##    (Intercept)        pressure            temp  
##      65.475357       -0.061323        3.398008  
##   locationLund  locationAbisko   pressure:temp  
##       0.343368       -0.483704       -0.003326
\end{verbatim}

\hypertarget{k}{%
\subsection{3 k)}\label{k}}

Sincer there are approximately the same number of observations for each
season there is no need to relevel. THe result of the forward selection
implies seasonal differences. The partial F-test shows that the final
model is a significant improvment on the previous best (2n) - p-value
\textless{} 1e-11! The final model now explains more than 42\% of the
variability.

\begin{Shaded}
\begin{Highlighting}[]
\NormalTok{weather.clean}\OperatorTok{$}\NormalTok{season <-}\StringTok{ "Winter"}
\NormalTok{weather.clean}\OperatorTok{$}\NormalTok{season[weather.clean}\OperatorTok{$}\NormalTok{monthnr }\OperatorTok{>}\StringTok{ }\DecValTok{2} \OperatorTok{&}\StringTok{ }\NormalTok{weather.clean}\OperatorTok{$}\NormalTok{monthnr }\OperatorTok{<}\StringTok{ }\DecValTok{6}\NormalTok{] <-}\StringTok{ "Spring"}
\NormalTok{weather.clean}\OperatorTok{$}\NormalTok{season[weather.clean}\OperatorTok{$}\NormalTok{monthnr }\OperatorTok{>}\StringTok{ }\DecValTok{5} \OperatorTok{&}\StringTok{ }\NormalTok{weather.clean}\OperatorTok{$}\NormalTok{monthnr }\OperatorTok{<}\StringTok{ }\DecValTok{9}\NormalTok{] <-}\StringTok{ "Summer"}
\NormalTok{weather.clean}\OperatorTok{$}\NormalTok{season[weather.clean}\OperatorTok{$}\NormalTok{monthnr }\OperatorTok{>}\StringTok{ }\DecValTok{8} \OperatorTok{&}\StringTok{ }\NormalTok{weather.clean}\OperatorTok{$}\NormalTok{monthnr }\OperatorTok{<}\StringTok{ }\DecValTok{12}\NormalTok{] <-}\StringTok{ "Fall"}
\NormalTok{weather.clean}\OperatorTok{$}\NormalTok{season <-}
\StringTok{  }\KeywordTok{factor}\NormalTok{(weather.clean}\OperatorTok{$}\NormalTok{season,}
         \DataTypeTok{levels =} \KeywordTok{c}\NormalTok{(}\StringTok{"Winter"}\NormalTok{, }\StringTok{"Spring"}\NormalTok{, }\StringTok{"Summer"}\NormalTok{, }\StringTok{"Fall"}\NormalTok{),}
         \DataTypeTok{labels =} \KeywordTok{c}\NormalTok{(}\StringTok{"Winter"}\NormalTok{, }\StringTok{"Spring"}\NormalTok{, }\StringTok{"Summer"}\NormalTok{, }\StringTok{"Fall"}\NormalTok{))}
\KeywordTok{summary}\NormalTok{(weather.clean}\OperatorTok{$}\NormalTok{season)}
\end{Highlighting}
\end{Shaded}

\begin{verbatim}
## Winter Spring Summer   Fall 
##    270    273    264    277
\end{verbatim}

\begin{Shaded}
\begin{Highlighting}[]
\KeywordTok{step}\NormalTok{(}\KeywordTok{lm}\NormalTok{(}\KeywordTok{log}\NormalTok{(rain)}\OperatorTok{~}\DecValTok{1}\NormalTok{, }\DataTypeTok{data =}\NormalTok{ weather.clean), }
     \DataTypeTok{scope =} \KeywordTok{list}\NormalTok{(}\DataTypeTok{upper =} \KeywordTok{lm}\NormalTok{(}\KeywordTok{log}\NormalTok{(rain)}\OperatorTok{~}\StringTok{ }\NormalTok{temp}\OperatorTok{*}\NormalTok{pressure}\OperatorTok{*}\NormalTok{location}\OperatorTok{*}\NormalTok{season, }\DataTypeTok{data =}\NormalTok{ weather.clean)), }
     \DataTypeTok{direction =} \StringTok{"forward"}\NormalTok{, }\DataTypeTok{k =} \KeywordTok{log}\NormalTok{(}\KeywordTok{nrow}\NormalTok{(weather.clean)))}
\end{Highlighting}
\end{Shaded}

\begin{verbatim}
## Start:  AIC=-648.5
## log(rain) ~ 1
## 
##            Df Sum of Sq    RSS     AIC
## + pressure  1    72.929 519.20 -783.99
## + season    3    71.430 520.70 -766.89
## + temp      1    61.245 530.88 -759.87
## + location  2    49.526 542.60 -729.21
## <none>                  592.13 -648.50
## 
## Step:  AIC=-783.99
## log(rain) ~ pressure
## 
##            Df Sum of Sq    RSS     AIC
## + temp      1    89.943 429.25 -983.22
## + location  2    90.522 428.67 -977.69
## + season    3    69.137 450.06 -917.93
## <none>                  519.20 -783.99
## 
## Step:  AIC=-983.22
## log(rain) ~ pressure + temp
## 
##                 Df Sum of Sq    RSS      AIC
## + location       2    54.324 374.93 -1115.92
## + temp:pressure  1    22.073 407.18 -1033.45
## + season         3    23.095 406.16 -1022.20
## <none>                       429.25  -983.22
## 
## Step:  AIC=-1115.92
## log(rain) ~ pressure + temp + location
## 
##                     Df Sum of Sq    RSS     AIC
## + temp:pressure      1   18.4915 356.44 -1163.8
## + season             3   21.8427 353.09 -1160.0
## <none>                           374.93 -1115.9
## + pressure:location  2    4.5080 370.42 -1115.0
## + temp:location      2    2.7292 372.20 -1109.9
## 
## Step:  AIC=-1163.75
## log(rain) ~ pressure + temp + location + pressure:temp
## 
##                     Df Sum of Sq    RSS     AIC
## + season             3   17.9906 338.45 -1198.9
## <none>                           356.44 -1163.8
## + temp:location      2    0.3441 356.09 -1150.8
## + pressure:location  2    0.2435 356.19 -1150.5
## 
## Step:  AIC=-1198.93
## log(rain) ~ pressure + temp + location + season + pressure:temp
## 
##                     Df Sum of Sq    RSS     AIC
## <none>                           338.45 -1198.9
## + temp:season        3    4.2638 334.18 -1191.7
## + pressure:location  2    0.6187 337.83 -1186.9
## + temp:location      2    0.2452 338.20 -1185.7
## + pressure:season    3    1.4178 337.03 -1182.5
## + location:season    6    6.7551 331.69 -1178.8
\end{verbatim}

\begin{verbatim}
## 
## Call:
## lm(formula = log(rain) ~ pressure + temp + location + season + 
##     pressure:temp, data = weather.clean)
## 
## Coefficients:
##    (Intercept)        pressure            temp  
##      62.404924       -0.058266        3.081894  
##   locationLund  locationAbisko    seasonSpring  
##       0.347835       -0.528447       -0.200369  
##   seasonSummer      seasonFall   pressure:temp  
##       0.132566        0.153757       -0.003021
\end{verbatim}

\begin{Shaded}
\begin{Highlighting}[]
\NormalTok{model.final <-}\StringTok{ }\KeywordTok{lm}\NormalTok{(}\KeywordTok{log}\NormalTok{(rain)}\OperatorTok{~}\StringTok{ }\NormalTok{temp}\OperatorTok{*}\NormalTok{pressure }\OperatorTok{+}\StringTok{ }\NormalTok{season }\OperatorTok{+}\StringTok{ }\NormalTok{location, }\DataTypeTok{data =}\NormalTok{ weather.clean)}
\KeywordTok{anova}\NormalTok{(model}\FloatTok{.2}\NormalTok{n.clean, model.final)}
\end{Highlighting}
\end{Shaded}

\begin{verbatim}
## Analysis of Variance Table
## 
## Model 1: log(rain) ~ temp * pressure + location
## Model 2: log(rain) ~ temp * pressure + season + location
##   Res.Df    RSS Df Sum of Sq      F    Pr(>F)    
## 1   1078 356.44                                  
## 2   1075 338.45  3    17.991 19.048 4.864e-12 ***
## ---
## Signif. codes:  
## 0 '***' 0.001 '**' 0.01 '*' 0.05 '.' 0.1 ' ' 1
\end{verbatim}

\begin{Shaded}
\begin{Highlighting}[]
\KeywordTok{summary}\NormalTok{(model.final)}\OperatorTok{$}\NormalTok{r.squared}
\end{Highlighting}
\end{Shaded}

\begin{verbatim}
## [1] 0.4284204
\end{verbatim}

\begin{Shaded}
\begin{Highlighting}[]
\KeywordTok{summary}\NormalTok{(model.final)}\OperatorTok{$}\NormalTok{adj.r.squared}
\end{Highlighting}
\end{Shaded}

\begin{verbatim}
## [1] 0.4241668
\end{verbatim}

\end{document}
